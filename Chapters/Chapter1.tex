\chapter{Introduction}
\label{chap:intro}
\lhead{\emph{Introduction}}
%This chapter should comprise around 1000 words and introduces your project. Here you are setting the scene, remember the reader may know nothing about your project at this stage (other than the abstract). N.B. The sections outlined in this document are suggested, some projects will have a greater or lesser emphasis on different sections or may change titles and some will have to add other sections to provide context or detail.
% Putting in comments within the TeX file can be really useful in making notes for yourself and dumping text that you intend to edit later

\section{Motivation}
%Why is it important to do a project on this topic? This should cover your key motivation for this. For example an excellent student from 2016 noticed a large number of homeless sleeping rough in Cork and was motivated to develop a system that load balanced the homeless shelters to try to accommodate the maximum number of homeless. This section can include the personal pronoun but the rest of the report should be third person passive, this is the case with most technical reports! For example here it is fine to say "... I decided to develop and app to help ...".
Access to the internet is becoming more and more easier as well as cheaper allowing more devices to communicate between each other. In order for those devices to communicate securely and only communicate with the parties chosen methods of encryption need to be exercised. Constant increase in computing performance has proven to make some encryption methods depreciated due to a high chance of successful decryption with little time involved. This increase can be easily expressed since the computer used to send the first rocket to the moon is less powerful than a cheap smart phone today. 

Industry standard encryption methods today rely on the fact that it would take longer than the lifetime of earth to crack an encryption key using today's computing power provided there are no vulnerabilities in the encryption method, and the attacker didn't get lucky and manage to guess the key early on in the attack.
The chance of randomly generating the correct key is very very low and therefore when cracking cryptography methods other forms of attacks are used. 
One of these methods could include having unauthorised access to a host and finding keys on the disk, cache or ram.

Discovery of a key by means of any method will lead to successfully decrypting interactions between hosts of a particular session since each session generally encrypts information with the same key.

Due to this, methods of encryption should acknowledge the possibility of the key being leaked or stolen and therefore minimise or ideally nullify the amount of potential private information available to the attacker with the stolen key.
This project will provide a solution to the above by using dynamic encryption to essentially switch keys during a session.

\section{Contribution}
%Enumerate the main contributions. Here try to zoom out, to talk from the perspective of a Computer Science graduate. In other words, imagine you are talking to a job panel, and you want to show your computer science skills by enumerating how they are reflected in your project work. A good guide here is to look back over the modules you have covered as an undergrad from 2/3rd year, how many tools and techniques from these modules do you have in the project and to what extent? How have you advanced beyond the module content? Do you have anything new?
The outcome of this project could benefit companies or individuals who want to transfer information in the most secure way possible. Dynamic encryption as of writing this paper is a niche topic primarily due to the fact that industry standard encryption methods are generally secure enough and quite difficult for attackers to overcome. There is a company called Dencrypt \cite{dencrypt} that heavily uses dynamic encryption for voip communication so perhaps the use of dynamic encryption will increase and be used by other companies in the future in order to transfer other types of information.

This research paper focuses on improving the computing area of secure transfer of data as well as cryptography with the help of machine learning algorithms. Using machine learning for cryptography is a topic of research undertaken by large companies such as Google, however the researchers mostly focused on creating new encryption algorithms whereas this paper adds additional measures to existing industry standard encryption methods. 


\section{Structure of This Document}
% notice how I cross referenced the chapters through using the \label tag --> LaTeX is VERY similar to HTML and other mark up languages so you should see nothing new here!
%This section is quite formulaic. Briefly describe the structure of this document, enumerating what does each chapter and section stands for. For instance in this work in Chapter \ref{chap:background} the guidance in structuring the literature review is given. Chapter \ref{chap:problem} describes the main requirements for the problem definition and so on ...
Chapter one of this document is the introduction to this paper. The motivation describes why I choose to do a project revolving around dynamic encryption and secure transfer of information. The contribution section describes how and why my project will impact companies and security minded individuals. The structure of this document section briefly describes the structure of this document and what each of the chapters and sections entail.

Chapter two describes the project background related to computer science. The main area in which the project falls under is explored, the main areas and topics which correspond to this project are identified. Thematic Area within Computer Science describes the technologies and methods that the project uses and builds upon. A Review of thematic depicts the research papers, articles, forums, blogs and anything else that has been used for research.

Chapter three explains the problem this project is solving including any objectives and desired achievements. Functional and non-functional requirements are also listed as a conclusion to the chapter.

Chapter four explains the implementation aproach...........