\chapter{Conclusions and Future Work}
\label{chap:conclusions}
\lhead{\emph{Conclusions}}
%This chapter should comprise 2-3 pages and enumerate conclusions of this phase of work. In your final report Discussions and Conclusions will form separate chapters and be significantly longer and more detailed.

\section{Discussion}
%A reflective discussion of some of the problems you encountered during this phase of the project and how that may influence how you proceed with the next phase.
This project uses a number of different libraries for different tasks. One of the most important tasks is getting the REST API to function properly. Initially I wanted to use the Cpprestsdk library as it was the most popular as well as being actively developed by Microsoft. This did not come to fruition due to the sample examples not compiling on my system after installing and linking the library during compiling. In order to solve this problem I began looking for alternative REST frameworks and stumbled upon Pistache which turned out to be more responsive than Cpprestsdk and even native PHP. This means that each request is processed in less time and more requests can be processed per second, which will reduce the time needed to synchronise the tree parity machines. Due to Pistache claiming to be written in pure C++11 with no external dependencies I was able to successfully compile the examples provided on their GitHub. 

Another problem that surfaced was due to my lack of knowledge about advanced C++ concepts while writing the prototype. I needed to use a generic data type, specifically the DynamicArray class needed to contain a generic variable and it needed a template function. I refreshed my knowledge on templates by watching a small number of tutorials such as this \cite{CppTotorial}. However I still had issues with linking during compilation. While trying to fix the issue I discovered \cite{cppHeaderCrap} that it is best to put function templates only in header files separating templates into header and definitions causes issues therefore The whole DynamicArray class now resides in a header file and has no corresponding .cpp file. I will have to reinforce my knowledge in other aspects of C++ as different challenges will arise throughout the project later on.

\section{Conclusion}
%Enumerate the main conclusions you have got in terms of background, problem description and the solution approach you have come up with.


%problem description
The problem this project attempts to solve has been clear from the very beginning. I knew that I wanted to work on an encryption system that was capable of changing keys throughout the session between connected hosts without obviously sending the keys through the network. My current solution sends information through the network that helps build these keys however by grabbing this information and attempting to construct the key from it in theory is impossible. My project aims to improve security over just using standard methods, It can be argued that standard encryption methods are generally good enough and this is true it is very difficult to the point of almost impossible to decrypt data without the key that was been encrypted using standard encryption. However my project aims to go the extra mile that even if the attacker managed somehow to steal the key, potentially only a small piece of the session could potentially be accessible to the attacker and that small piece in relation to the whole session would be meaningless and any sensitive information should be safe.

%solution approach
The way I intended to achieve dynamic encryption varied a number of times. Initially I had difficulty in coming up with a reasonable way to achieve this. The difficulty originated due to the fact that the key should not be sent to the other host directly or at least in an obvious manner. Before finding out about the existence of tree parity machines I was going to base my encryption method to the one used by Dencrypt where the encrypted packets are layered with an unknown encryption method, this would mean that i would need to research many encryption methods and some way to detect or at least have the other host detect which encryption method was used since how Dencrypt achieved dynamic encryption was secret and they only provided a brief overview. I had no idea how to safely detect the encryption method without brute forcing every possible method which would be useless to begin with as an attacker would be able to do the same. My next idea was to have the initial host generate a large number of keys within which the real key would be held and have the host intelligently select and test potential keys using some sort of a trained AI algorithm, my reasoning behind this would be that the attacker would need to test each key which would take too long whereas the host with the AI engine would only need to test a relatively small number. However for this to be feasible the encryption technique must be relatively slow reducing the number of keys tested per second to a feasible minimum. And the file sent over to the other host with the keys would need to be tens of GigaBytes large and only contain one correct key, therefore making this idea useless. I needed some sort of way that allowed for the same data to be present on both hosts without specifically sending this data, magic I believed before stumbling upon synchronising tree parity machines. I found the idea of two neural networks learning from each other that eventually leads to them having the same weights fascinating. I knew then that this is how I will accomplish my project. On top of the tree parity machines I would need to have some sort of network interface that would allow the "syncing data" to travel to other hosts. And to make the dynamic encryption more dynamic there will be multiple tree parity machines synchronising which i believe would lead to a steady supply of secret keys. The diagrams provided earlier in this paper would suffice for achieving this system although I do fear that due to the many packets needed to synchronise a single tree parity machine never mind multiple will put a strain on the application which can be solved by running multiple instances which might surface issues regarding that, this however will be something to deal with in semester two.


\section{Future Work}
%Enumerate all the things you would have wanted to do should you have more time to work on this report.
Given the limited time available for actual code in semester one due to research among other things being the focus, the prototype ended up being very basic. There was no time to do anything worth showing using Pistache the REST framework. Ideally a simple server could have been setup with the desired routes and basic functionality that allows data to be bounced between two instances of the server. This could mimic the queries sent back and forward between the tree parity machines, since those queries are going to be sent using TCP or more specifically HTTPS this would give me a good start for next semester as it would be easy to replace meaningless data with query data.
The prototype uses random data for Tree Parity Machine input as it was quicker and easier to implement however ideally queries should have been used for security purposes and will definitely make an appearance in the final solution.

%Additional resources on the use of latex is below.

%Tutorials:
%\begin{itemize}
%    \item \url{https://www.latex-tutorial.com/tutorials/beginners/how-to-use-latex}
%    \item \url{https://en.wikibooks.org/wiki/LaTeX}
%    \item \url{https://www.sharelatex.com/learn/Main_Page}
%    \item \url{http://www.math.harvard.edu/texman}
%    \item \url{https://web.stevens.edu/hfslwiki/images/a/a0/ShareLatex_Tutorial.pdf}
%\end{itemize}

%Presentations:
%\begin{itemize}
%    \item \url{http://www.iu.hio.no/~frodes/rm/ppt/latex.ppt}
%    \item \url{https://classes.soe.ucsc.edu/ams200/Fall09/Latex_intro.ppt}
%    \item \url{http://www.menet.umn.edu/~blake/latexcourse/courseslides.ppt}
%\end{itemize}
