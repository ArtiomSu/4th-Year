\chapter{DynamiCrypt}
\label{chap:problem}
\lhead{\emph{Problem Statement}}
%The key question to be addressed in this chapter is: "What do I want to achieve".

%This chapter should comprise around 1500 words and describe the problem you are trying to solve. Try to be as specific here as you can, this will help you to anticipate possible risks such as lack of support from APIs.

\section{Problem Definition}
%Describe the problem you are trying to solve in this project. There will sometimes be a need at some point during the report to display an equation that may be core to your project. For example if the project is on gait detection what equation are you using to determine gait? If the project is on localization what is the method/formula? The formatting of these is reliably done in Latex also as we can see in equation \ref{eq:Legrange}.

%\begin{equation}
%\frac{d}{dt}(\frac{\partial L}{\partial \dot{c_i}})-\frac{\partial L}{\partial %c_i}+\frac{\partial P}{\partial \dot{c_i}} = F_i,
%\label{eq:Legrange}
%\end{equation}
The problem this project is designed to solve is increase the level of security when transferring data between hosts. Traditionally when transferring data between two hosts an encryption key for that session is generated and normally used until the session ends. In some cases the same key is used for future communications. This can cause issues and potential data theft if the attacker manages to capture the session and get access to the key by any means necessary, it would be possible to decrypt the session and potentially reveal important/private information.

The information revealed can potentially cause data breaches. In 2018 alone major data breaches have occurred leaking millions of private user data including Facebook where 87 million users were compromised \cite{Data_breach_facebook}. The largest data breach this year affected 1.1 billion registered Indian citizens. The company that was compromised is Aadhaar \cite{Data_breach_Aadhaar} causing personal addresses, phone numbers, emails and names to be available to anyone willing to part with a few hundred rupees.

This project will mitigate this potential problem by periodically changing the encryption key throughout the duration of the session with both parties being aware of the current key in use while the attacker will not know the key. By using this method if the attacker manages to get the encryption key only a small portion of the session could be decrypted. The random keys generated for encryption will be isolated from each other therefore if the attacker has one key it will not be possible to perform mathematical operations on the key to calculate the next keys used for encryption. 

\section{Objectives}
%Enumerate the objectives you want to achieve in your project. Again as this is an early stage these will tend to change but there should be a rational explanation for this change. Always document your work, keep a lab book during the term that you only use for FYP!
The objectives of this project focus on providing a sort of an API that allows to dynamically synchronises with other remote hosts and shares encryption keys with other servers running on the same machine, in order to provide dynamically encrypted information.

Objectives are identified in a way that makes then achievable under the time constraint.

\begin{enumerate}
	\item Provide an API that can be configured only by servers running only on the same machine.
	\item Provide a number of tree parity machines that will synchronise with remote tree parity machines these will be configured through the API and will be able to access machines outside of localhost.
	\item Provide methods of identification for threads, hosts and threads for the different hosts.
	\item Provide methods for allowing multiple hosts to access the API.
	\item Evaluate performance and maybe use and test different learning algorithms.
	\item Create an easy to use NodeJs module for easy implementation with NodeJs Apps.
\end{enumerate}

\section{Functional Requirements}
%Enumerate the functional requirements you want your project to have. 

%Please, do not include the use cases here. If you want to create a one-to-one mapping between functional requirements and use cases (which does not necessarily need to be the case, indeed most likely this will not be the case) do it elsewhere. Here should purely describe what do you want to do. In no case should you use this section to provide a description of how to implement them, that is for later. For people doing projects that are not heavy implementation projects (e.g. deploying an architecture or testing a novel tool in specific conditions) this structure can still be used as it will force you to think about what you plan to achieve and what possible metrics you may need to measure success.

%Let me explain this with more detail. A common mistake is that people confuse the problem description with the solution approach. This is a common mistake by confusing the \emph{what} with the \emph{how}. Here we are purely focused on the what: What is this project about? What are the objectives? What are the functional and non-functional requirements? 

%How are we going to do all these things? Well, this is a question for next chapter. Provided a problem, an objective or a functional requirement, obviously there will usually be many ways of doing it, thus there will be many \emph{hows}, but the definition, the \emph{what} we want to achieve will be unique.

%One other display structure you may wish to use at some stage during the report is a figure array. This can also be easily done with Latex and is shown in figure \ref{fig:twosuccesskid}

%\begin{figure}
%\centering     %%% not \center
%\subfigure[Figure A]{\label{fig:a}\includegraphics[width=0.48\textwidth]{successkid.jpg}}
%\subfigure[Figure B]{\label{fig:b}\includegraphics[width=0.48\textwidth]{successkid.jpg}}
%\caption{Two Success kids}
%\label{fig:twosuccesskid}
%\end{figure}
\begin{enumerate}

	\item Share connection information between hosts. Port number(s), number of synchronising threads, thread id(s) and maybe more.
	\item Threads should be easily identified as there will be multiple threads synchronising simultaneously. 
	\item Each thread synchronises with remote thread.
	\item API sends one key to another server running on localhost and only servers running on localhost after synchronisation between one or more threads occurred.
	\item After key is extracted from thread the localhost thread and corresponding remote thread will desynchronise and begin to synchronise again.
	\item API will be notified when a host disconnects and connects to clean up accordingly.
	\item sensitive information in ram should be protected/encrypted.
	\item sensitive information will not be saved in logs or cache.
	
\end{enumerate}




\section{Non-Functional Requirements}
%Enumerate the non-functional requirements you want to achieve in your project (i.e. broadly speaking how your system will operate).
\begin{enumerate}

	\item The first thread to synchronise should take no longer than 2 seconds.
	\item The program should work on any Linux host with a kernel version of at least 3.10.
	\item The NodeJs module should support NodeJs version 8 and above. 
	\item The NodeJs module should provide easy to use callbacks and hide the complexity.
	\item The install script should work on all supported distros provided dependencies are met. 
	\item The API should require authentication.
	\item The synchronising threads should have some sort of digital signature or a method of identification.
	\item The API should support the ability to synchronise with multiple hosts at the same time or multiple instances of the API can be executed or some form of master deals with multiple instances. Perhaps a different port can be used for each connecting host.
	\item Program should run under its own user to minimise tampering.
\end{enumerate}


