%% ----------------------------------------------------------------
%% Thesis.tex -- MAIN FILE (the one that you compile with LaTeX)
%% ---------------------------------------------------------------- 

% Set up the document
\documentclass[a4paper, 11pt, oneside]{Thesis}  % Use the "Thesis" style, based on the ECS Thesis style by Steve Gunn

% Include any extra LaTeX packages required
\usepackage[square, numbers, comma, sort&compress]{natbib}  % Use the "Natbib" style for the references in the Bibliography
\usepackage[nottoc]{tocbibind} % bind bibliography to the table of contents
\usepackage{verbatim}  % Needed for the "comment" environment to make LaTeX comments
\usepackage{vector}  % Allows "\bvec{}" and "\buvec{}" for "blackboard" style bold vectors in maths
\usepackage[table]{xcolor}
\hypersetup{urlcolor=black, colorlinks=true}  % Colours hyperlinks in black, can be distracting if there are many links and colored blue.
\usepackage{graphicx}
\graphicspath{{Figures/}}  % Location of the graphics files (set up for graphics to be in PDF format)

%% ----------------------------------------------------------------
\begin{document}
\frontmatter      % Begin Roman style (i, ii, iii, iv...) page numbering

% Set up the Title Page
\title  {DynamiCrypt}
\authors  {Artiom Sumigora}
            
\addresses  {\groupname\\\deptname\\\univname}  % Do not change this here, instead these must be set in the "Thesis.cls" file, please look through it instead
\date       {\today}
\subject    {}
\keywords   {}

\maketitle
%% ----------------------------------------------------------------

\setstretch{1.3}  % It is better to have smaller font and larger line spacing than the other way round

% Define the page headers using the FancyHdr package and set up for one-sided printing
\fancyhead{}  % Clears all page headers and footers
\rhead{\thepage}  % Sets the right side header to show the page number
\lhead{}  % Clears the left side page header

\pagestyle{fancy}  % Finally, use the "fancy" page style to implement the FancyHdr headers

%% ----------------------------------------------------------------
% Declaration Page required for the Thesis
\Declaration{

\addtocontents{toc}{\vspace{1em}}  % Add a gap in the Contents, for aesthetics

I, Artiom Sumigora, declare that this thesis titled, `DynamiCrypt' and the work presented in it are my own. I confirm that:

\begin{itemize} 
\item[\tiny{$\blacksquare$}] This work was done wholly or mainly while in candidature for an undergraduate degree at Cork Institute of Technology.
 
\item[\tiny{$\blacksquare$}] Where any part of this thesis has previously been submitted for a degree or any other qualification at Cork Institute of Technology or any other institution, this has been clearly stated.
 
\item[\tiny{$\blacksquare$}] Where I have consulted the published work of others, this is always clearly attributed.
 
\item[\tiny{$\blacksquare$}] Where I have quoted from the work of others, the source is always given. With the exception of such quotations, this project report is entirely my own work.
 
\item[\tiny{$\blacksquare$}] I have acknowledged all main sources of help.
 
\item[\tiny{$\blacksquare$}] Where the thesis is based on work done by myself jointly with others, I have made clear exactly what was done by others and what I have contributed myself.
\\
\end{itemize}
 
 
Signed:\\
\rule[1em]{25em}{0.5pt}  % This prints a line for the signature
 
Date:\\
\rule[1em]{25em}{0.5pt}  % This prints a line to write the date
}
\clearpage  % Declaration ended, now start a new page

%% ----------------------------------------------------------------

% The Abstract Page
\addtotoc{Abstract}  % Add the "Abstract" page entry to the Contents
\abstract{
\addtocontents{toc}{\vspace{1em}}  % Add a gap in the Contents, for aesthetics

In today's world information is mostly sent in an encrypted form over the public internet. Traditionally when a client connects to a server the public keys are shared and the same set of public/private key pairs are used for the session and potentially for future sessions, depending on how the system is setup. Provided that industry standard encryption is used it would take the attacker longer than the lifetime of the earth to crack the key making the system secure. The problem arises if the attacker managed to get the key in some other fashion other than brute force and got access to the server it would be possible to locate this key and use it to decrypt potentially sensitive information that was captured over the network.

The problem this project proposes to address is that scenario. This will be addressed by using multiple encryption keys throughout the session. These encryption keys will be generated using a tree parity machine. A tree parity machine consists of input neurons, hidden neurons and one output neuron. A neural network is chosen for this because the weights of neural networks can be synchronised between each tree parity machine on different hosts. The weights can then be used to generate a key and since the weights on both tree parity machines are identical the same key will be generated. The weight are synchronised over the network with no information sent about the weights itself therefore the attacker will not be able to figure out the key. The server can then use multiple tree parity machines to synchronise with the same number of tree parity machines on a different server and thus multiple keys will be generated and different keys can be used through a session.

This thesis will provide a program that is capable of handling synchronisation between multiple tree parity machines between two hosts. An API will be provided that will access the said program and allow other server to use dynamic encryption. A NodeJS module will also be provided to make integration with NodeJS applications very easy.

}

\clearpage  % Abstract ended, start a new page
%% ----------------------------------------------------------------

\setstretch{1.3}  % Reset the line-spacing to 1.3 for body text (if it has changed)

% The Acknowledgements page, for thanking everyone
\acknowledgements{
\addtocontents{toc}{\vspace{1em}}  % Add a gap in the Contents, for aesthetics
This project has taken a substantial amount of work, dedication and research. I would
like to say special thanks to:

Dr. John Creagh, project supervisor in semester one.

}
\clearpage  % End of the Acknowledgements
%% ----------------------------------------------------------------

\pagestyle{fancy}  %The page style headers have been "empty" all this time, now use the "fancy" headers as defined before to bring them back


%% ----------------------------------------------------------------
\lhead{\emph{Contents}}  % Set the left side page header to "Contents"
\tableofcontents  % Write out the Table of Contents

%% ----------------------------------------------------------------
\lhead{\emph{List of Figures}}  % Set the left side page header to "List if Figures"
\listoffigures  % Write out the List of Figures

%% ----------------------------------------------------------------
\lhead{\emph{List of Tables}}  % Set the left side page header to "List of Tables"
\listoftables  % Write out the List of Tables

%% ----------------------------------------------------------------
\setstretch{1.5}  % Set the line spacing to 1.5, this makes the following tables easier to read
\clearpage  % Start a new page
\lhead{\emph{Abbreviations}}  % Set the left side page header to "Abbreviations"
\listofsymbols{ll}  % Include a list of Abbreviations (a table of two columns)
{
% \textbf{Acronym} & \textbf{W}hat (it) \textbf{S}tands \textbf{F}or \\
\textbf{TPM} & \textbf{T}ree \textbf{P}arity \textbf{M}achine \\
\textbf{API} & \textbf{A}pplication \textbf{P}rotocol \textbf{I}nterface \\
\textbf{HTTP} & \textbf{H}yper \textbf{T}ext \textbf{T}ransfer \textbf{P}rotocol \\
\textbf{HTTPS} & \textbf{H}yper \textbf{T}ext \textbf{T}ransfer \textbf{P}rotocol \textbf{S}ecure \\
\textbf{SSL} & \textbf{S}ecure \textbf{S}ockets \textbf{L}ayer \\
\textbf{TLS} & \textbf{T}ransfer \textbf{L}ayer \textbf{S}ecurity \\
\textbf{AES} & \textbf{A}dvanced \textbf{E}ncryption \textbf{S}tandard \\
\textbf{RSA} & \textbf{R}ivest \textbf{S}hamir \textbf{A}dleman \\
\textbf{SSH} & \textbf{S}ecure \textbf{S}\textbf{H}ell \\
\textbf{GPG} & \textbf{G}NU \textbf{P}rivacy \textbf{G}uard \\
\textbf{GNU} & \textbf{G}\textbf{N}\textbf{U}'s Not Unix \\
\textbf{AI} & \textbf{A}rtificial \textbf{I}ntelligence \\
\textbf{GAN} & \textbf{G}enerative \textbf{A}dversarial \textbf{N}etwork \\
\textbf{IP} & \textbf{I}nternet \textbf{P}rotocol \\
\textbf{IBM} & \textbf{I}nternational \textbf{B}usiness \textbf{M}achines \\
\textbf{ID} & \textbf{I}\textbf{D}entification \\
\textbf{PHP} & \textbf{H}ypertext \textbf{P}re\textbf{P}rocessor \\
\textbf{KiB} & \textbf{K}\textbf{i}bi\textbf{B}yte \\
\textbf{TCP} & \textbf{T}ransmission \textbf{C}ontrol \textbf{P}rotocol \\
}

%% ----------------------------------------------------------------
% End of the pre-able, contents and lists of things
% Begin the Dedication page

\setstretch{1.3}  % Return the line spacing back to 1.3

\pagestyle{empty}  % Page style needs to be empty for this page
\dedicatory{Dedicated to my family\ldots}

\addtocontents{toc}{\vspace{2em}}  % Add a gap in the Contents, for aesthetics

%% ----------------------------------------------------------------
\mainmatter	  % Begin normal, numeric (1,2,3...) page numbering
\pagestyle{fancy}  % Return the page headers back to the "fancy" style

\chapter{Introduction}
\label{chap:intro}
\lhead{\emph{Introduction}}
%This chapter should comprise around 1000 words and introduces your project. Here you are setting the scene, remember the reader may know nothing about your project at this stage (other than the abstract). N.B. The sections outlined in this document are suggested, some projects will have a greater or lesser emphasis on different sections or may change titles and some will have to add other sections to provide context or detail.
% Putting in comments within the TeX file can be really useful in making notes for yourself and dumping text that you intend to edit later

\section{Motivation}
%Why is it important to do a project on this topic? This should cover your key motivation for this. For example an excellent student from 2016 noticed a large number of homeless sleeping rough in Cork and was motivated to develop a system that load balanced the homeless shelters to try to accommodate the maximum number of homeless. This section can include the personal pronoun but the rest of the report should be third person passive, this is the case with most technical reports! For example here it is fine to say "... I decided to develop and app to help ...".
Access to the internet is becoming more and more easier as well as cheaper allowing more devices to communicate between each other. In order for those devices to communicate securely and only communicate with the parties chosen methods of encryption need to be exercised. Constant increase in computing performance has proven to make some encryption methods depreciated due to a high chance of successful decryption with little time involved. This increase can be easily expressed since the computer used to send the first rocket to the moon is less powerful than a cheap smart phone today. 

Industry standard encryption methods today rely on the fact that it would take longer than the lifetime of earth to crack an encryption key using today's computing power provided there are no vulnerabilities in the encryption method, and the attacker didn't get lucky and manage to guess the key early on in the attack.
The chance of randomly generating the correct key is very very low and therefore when cracking cryptography methods other forms of attacks are used. 
One of these methods could include having unauthorised access to a host and finding keys on the disk, cache or ram.

Discovery of a key by means of any method will lead to successfully decrypting interactions between hosts of a particular session since each session generally encrypts information with the same key.

Due to this, methods of encryption should acknowledge the possibility of the key being leaked or stolen and therefore minimise or ideally nullify the amount of potential private information available to the attacker with the stolen key.
This project will provide a solution to the above by using dynamic encryption to essentially switch keys during a session.

\section{Contribution}
%Enumerate the main contributions. Here try to zoom out, to talk from the perspective of a Computer Science graduate. In other words, imagine you are talking to a job panel, and you want to show your computer science skills by enumerating how they are reflected in your project work. A good guide here is to look back over the modules you have covered as an undergrad from 2/3rd year, how many tools and techniques from these modules do you have in the project and to what extent? How have you advanced beyond the module content? Do you have anything new?
The outcome of this project could benefit companies or individuals who want to transfer information in the most secure way possible. Dynamic encryption as of writing this paper is a niche topic primarily due to the fact that industry standard encryption methods are generally secure enough and quite difficult for attackers to overcome. There is a company called Dencrypt \cite{dencrypt} that heavily uses dynamic encryption for voip communication so perhaps the use of dynamic encryption will increase and be used by other companies in the future in order to transfer other types of information.

This research paper focuses on improving the computing area of secure transfer of data as well as cryptography with the help of machine learning algorithms. Using machine learning for cryptography is a topic of research undertaken by large companies such as Google, however the researchers mostly focused on creating new encryption algorithms whereas this paper adds additional measures to existing industry standard encryption methods. 


\section{Structure of This Document}
% notice how I cross referenced the chapters through using the \label tag --> LaTeX is VERY similar to HTML and other mark up languages so you should see nothing new here!
%This section is quite formulaic. Briefly describe the structure of this document, enumerating what does each chapter and section stands for. For instance in this work in Chapter \ref{chap:background} the guidance in structuring the literature review is given. Chapter \ref{chap:problem} describes the main requirements for the problem definition and so on ...
Chapter one of this document is the introduction to this paper. The motivation describes why I choose to do a project revolving around dynamic encryption and secure transfer of information. The contribution section describes how and why my project will impact companies and security minded individuals. The structure of this document section briefly describes the structure of this document and what each of the chapters and sections entail.

Chapter two describes the project background related to computer science. The main area in which the project falls under is explored, the main areas and topics which correspond to this project are identified. Thematic Area within Computer Science describes the technologies and methods that the project uses and builds upon. A Review of thematic depicts the research papers, articles, forums, blogs and anything else that has been used for research.

Chapter three explains the problem this project is solving including any objectives and desired achievements. Functional and non-functional requirements are also listed as a conclusion to the chapter.

Chapter four explains the implementation aproach........... % Introduction

\chapter{Background}
\label{chap:background}
\lhead{\emph{Background}}
The key question to answer in this chapter is: "What has been done/is being done". 

This chapter comprises around 4000 words and should put your project into context within Computer Science. Your focus here should be on the final section "Current State of the Art". This should be at least 2500 of the 4000 words of this section.

\section{Thematic Area within Computer Science}
The Core topic of this project is safely and dynamically encrypting messages between two parties. The communication will rely on multiple functioning NodeJs servers for transfer of encrypted messages. 

The core areas under which my project falls under is cryptography, security for encrypting and securing information. Machine learning will be used for establishing methods of secure information exchange. And finally networking due to the setup required of communicating between different servers and sending encrypted information.

Encryption \cite{encryptionDefinition} is when the plaintext of any form of data that can be easily read is converted to an unreadable encoded version. In order to retrieve the original data for viewing or processing it must be decoded using a specific algorithm and more than likely some sort of key, usually a lengthy password. Encryption may be used for encrypting files and operating systems on a user's hard drive. In today's world encryption is used religiously for data transferred between networks. Sensitive information like user's credentials are constantly being sent from the browser to the server when logging into websites for personalised content. The same is true for for even more high risk information like banking details, scans of identification documents and even keys. Websites that wish to be secure are now using HTTPS instead of HTTP. The Number of websites using HTTPS is constantly increasing see figure \ref{fig:httpsRise}

\begin{figure}[ht]
  \centering
      \includegraphics[width=0.9\textwidth]{Figures/httpsRise.png}
  \caption[A graph of HTTPS usage increase]{A graph of HTTPS usage increase\cite{https}}
  \label{fig:httpsRise}
\end{figure}

HTTP is not secure because information transmitted is in plaintext by default and extra steps are needed to encrypt the data. Because the author of the server can choose how the data is encrypted, it can lead to the theft of data as the implementation may not be correct or a weak algorithm is used.
On the other hand if a website uses HTTPS which is a common defined standard there will be minimal data theft see figure \ref{fig:https1}. HTTPS uses SSL or TLS which are protocols that use asymmetric keys (will be discussed later). SSL is generally used more often as it requires the server to acquire an SSL certificate from a trusted third party.

\begin{figure}[ht]
  \centering
      \includegraphics[width=0.6\textwidth]{Figures/httpsVsHttp.png}
  \caption[HTTP vs HTTPS]{HTTP vs HTTPS\cite{https1}}
  \label{fig:https1}
\end{figure}



Traditionally there are two encryption types.
\begin{enumerate}
    \item Symmetric
    \item Asymmetric
\end{enumerate}

Symmetric encryption uses the same key to encrypt and decrypt information. This type of encryption is usually used to encrypt information provided by a human generated key. It is not safe to send this key over a network as it can be stolen or the destination being sent to can be spoofed. There are multiple implementations of symmetric ciphers, the most common being AES, Twofish and Serpent. To increase your security at the cost of encryption and decryption time you can chain multiple ciphers together AES(Twofish(Key)).

Asymmetric or commonly known as public key encryption methods are commonly used for sharing data between between computers on different networks. This is because a set of keys are generated one being private and the other public. The private key is never shared and remains on the host that generated it. The public key on the other hand can be shared with the party you want to communicate securely with. The public key is used to encrypt the data that is about to be sent back. This data can only be decrypted using the private key. Therefore you can share your public key with anyone and they wont be able to decrypt messages sent from another host who used the same public key. The most common algorithm is RSA. Certain protocols also use public key algorithms like SSH for secure remote connections to foreign hosts. And GPG for verification of packages on Linux systems and an alternative over https for Github.

Protocols like SSH create a set of keys during the start of the session and those keys remain constant therefore if the private key was leaked the whole conversation could be decrypted if the packets have been captured and stored.  

Encryption in its general form is simply a mathematical algorithm that takes plaintext and combines it with some sort of key over a number of iterations eventually producing the ciphertext. This might entice some people to try and break those ciphers and recover the original plain text. Quite a number of attacks do exist.
Private keys are sometimes stored on the disk or in temporary files that are saved by programs during their execution, until reboot or they are cleared after a number of days. The attacker may be able to access the server physically or remotely using an unrelated exploit and copy the key.

Social engineering is an attack where a human pretends to be of an authority figure and convinces an unaware human to give up the key. This can be done by an attacker pretending to be an executive engineer in a company and convince the victim indirectly or directly to give up the keys by running obfuscated commands in the terminal which then send over the key to the attackers server.

If the key used is created by a human and not some sort of machine generator there are a few number attacks that can be performed that would not be feasible or possible if the key was generated or quite long. These attacks include brute force which creates keys in usually ascending order or based on some algorithm to increase the chance of success. Brute force will eventually try every key possible however even a small sized key of 12 characters containing numbers, symbols, upper and lover case numbers it would take around 200 years \cite{brute}. 
Dictionary attacks can be used if the key is part of a large dictionary of human created passwords. 

Attacks on proper keys that are generated by machines are more sophisticated and rely on cracking the algorithm or device used for encryption more so then the key.
Linear cryptanalysis \cite{cipher-attacks} is a plaintext attack which means that the attack can use any plain text they want and receive the ciphertext for it after putting it through a system. This attack uses linear approximations to describe the behaviour of the block cipher. After large number of pairs of plaintext and cipher text there is a possibility to learn something about the key.

Algebraic attacks \cite{cipher-attacks} can be used if the ciphers exhibit a high probability of a mathematical structure. 

Reverse Engineering \cite{cipher-attacks} can be used to either examine the source code of the algorithm or disassemble the binary which uses the algorithm to look as the assembly code of the algorithm.
Machine key generators usually use some form of a random number generator which are algorithms that usually take in a seed hopefully something that isn't the current time but that has been known to be used and return a key. Attacks can be made on this number generator if the seed is something predictable or the generator generates predictable numbers. 

If the device on which the algorithm is performing on is an embedded device you can perform side channel attacks \cite{cipher-attacks} where you measure the spikes and frequency of the power consumption when the encryption is taking place. 

%
%
% End of encription 
%
%

Machine learning has been known to be used for secure communications although it is not clear if it is used in servers with valuable data. Google's AI successfully created secure algorithms \cite{GoogleAi1} that use inhuman cryptographic schemes making them harder to crack. This technique is called GAN Cryptography \cite{GoogleAi2} for which a research paper can be found.

Currently dynamic encryption is exercised in voip phone calls by a company known as Dencrypt \cite{dencrypt} according to the explanation of their proprietary algorithm they use a wrapper on top of AES-256 which is a chosen algorithm that is discarded when the data transfer is finished.




This project will be compatible with NodeJs because it is one of the fastest growing server platform \cite{NodeJs} that can be easily set up and supports a large amount of modules.




Position your topic within Computer Science. This activity will aid you in your literature review also. We zoom out to see three levels:

% notice the enumerate structure to create itemized lists
\begin{enumerate}
    \item What is the core topic your project is about? e.g., Mobile app for online voting.
    \item What core area(s) does the project fall under? e.g., Mobile applications, Social Networking, Service Providers. 
    \item What main area(s) of Computer Science does the project fall under? e.g. Software Development, Cloud Computing.
\end{enumerate}

The ACM Computing Classification System (http://www.acm.org/about/class) will aid you in this, use the 2012 categories. Make sure to use figures and illustrations were appropriate. LaTeX will take care of the formatting of these. Do not try to get fancy here, you should concentrate on the content and not the formatting, this is why we are specifying LaTeX.

% Again take note of the structure, simply copy and paste this for future single figures
\begin{figure}[ht]
  \centering
      \includegraphics[width=0.7\textwidth]{successkid.jpg}
  \caption[A picture of the success kid!]{A picture of the success kid!\cite{Reference1}}
  \label{fig:successkid}
\end{figure}

You can specify the width and label for a figure which allows you to reference the figure and you can attribute a source in the figure caption as is done for figure \ref{fig:successkid}. Make sure you reference all external figures (i.e. figures you did not create yourself). Also use references for all figures e.g. use "... in figure \ref{fig:successkid} ..." NOT "... in the figure above ...".

%\section{Project Scope}
%Project specifics: Background minimum knowledge.

%Imagine you wanted to explain the specifics of your project to a person that knows nothing of Computer Science. You cannot talk about everything (as the idea is not to write a 500+ page report). Remember the reader at this stage can only be assumed to know what you have covered, so identify what are the minimum concepts belonging to the main areas (listed as 3 in the section before) and the core areas (listed as 2 in the section before) that you would need to explain so that the reader is able to understand the specifics of your project and indeed the following section. For example the minimum amount of knowledge about software development, cloud computing, mobile applications, social networking and service providers that are required so as to understand the specifics of a project about a mobile app for an on-line voting system. Here we are making the same trip we did before, but now in the opposite direction. Start zooming in from 3, then to 2 and finally to reach your project 1. Once the reader is finished this section they should be able to understand the proceeding sections (and have context for it within the project).

\section{A Review of -INSERT THEMATIC AREA-}
The focus of this section is at the heart of the project research phase. You must identify the main sources of information you should be aware of within your chosen area and pay regular attention to so as to strengthen your knowledge in the core topic you are working at. So here you should develop an knowledge of not only your core topic but also about the area of computer science the topic falls under. More specifically you should research the following:
\begin{itemize}
    \item The top 5 International Conferences and Journals most related to your topic. This is crucial, as it represents the main source for keeping you aware of what the state-of-the-art in your topic is.
    \begin{itemize}
        \item In particular it will make you aware of what other projects related to yours have been already done (so that you can compare/position your project w.r.t. these).
        \item What new techniques are being developed, so that you can apply them in your work. e.g. new frameworks for data visualization
    \end{itemize}
    \item The top 3 most recent books/texts related to your topic. There are many free resources from which you may download a relevant text on the topic of your project. Try to either download or borrow 3 recent (no older than 10 years) texts relating to the topic your project is on which you will use throughout the project as reference material and to aid in tackling a number of the technical problems you may encounter. Any PhD/MSc thesis that have published in the last 5 years relating to the topic are also invaluable resources as they will contain a state of the art and references in your project topic. Approach these only after reading/viewing the wikis/Youtube videos you find as a certain level of knowledge will be assumed about the topic.
    \item The top 5 companies/organizations potentially interested in the product you are developing. Finally, this is also crucial, as it forces you extend to purely programmer view of the project to a wider view considering the market, potential stakeholders and niches where your product can become useful. Moreover, Computer Science is a huge topic with loads of different works and roles. If you pick a project in the area you feel passionate about, and you identify what the market in this area is about, then you can drive your future professional career (from the very beginning) towards the path that makes you happier. I know that this does sound as a very technical reason, but I suppose we all agree is probably the most important of all reasons for choosing a particular project focus. 
    \item The top 5 wiki/forums/blogs/Youtube channels most related to your topic. This is crucial to you as well, as it represents a more accessible, personal and less informal way of communication with people working/interested on the same topic as you are. This communication is extremely helpful for improving your skills, solving potential doubts and increase the interest/relevance of the topic/area itself.
\end{itemize}

You should begin your journey of discovery in reverse order to the listing above (which is given in order of academic importance/significance). So when you are researching your topic first look up some TedX talks or youtube tutorials, then research what companies are doing in the area, then get a handful of very good texts on the core topics of your area (anything older than 5 years usually is not helpful here) and finally start reading conference or journal papers (again newer is better here). In particular during this section you may need to use tables to list resources. These are also automatically formatted in latex thus allowing you to concentrate on content. for example table \ref{tab:Mylar}.

\begin{table}[ht]
	\centering
		\begin{tabular}{ c  c  }
		\hline
		\hline
		Parameter & PET \\
		\hline
		Youngs Modulus & 2800-3100MPa \\
		Tensile Strength & 55-75MPa \\
		Glass Temperature & 75$^\circ$C \\
		Density & 1400kg/m$^3$ \\
		Thermal Conductivity & 0.15-0.24Wm$^{-1}$K$^{-1}$ \\
		Linear Expansion Coefficient & $7\times10^-5$ \\
		Relative Dielectric Constant @ 1MHz & 3\\
		Dielectric Breakdown Strength & 17kVmm$^{-1}$\\
		\end{tabular}
	\caption{PET Physical Properties}
	\label{tab:Mylar}
\end{table}

What has been done before in your community w.r.t. your topic? Once you have gotten an understanding of the topic and technologies and have identified the top 5 formal conferences/journals, wiki/forums/blogs/Youtube channels and companies/organizations the next step is to research in depth on them! And here in depth means in depth. Make sure you cite\cite{Reference1} a number of papers \cite{Reference3}, luckily Latex will take care of the ordering of the citations \cite{Reference2} for you.

The aim here is that you find the trends in your topic (3), and more in general in the area in which your topic resides (2) your project falls under and from these trends you develop your initial project question further and begin to get insights into how others have solved/approached similar problems. Think of this section as colouring in your initial idea. Before you approach this section you should read at least 4/5 good literature reviews (a selection of last years projects will be posted on blackboard to aid you but you should find other sources also).

In particular in this section, you must find and analyze at least 5 (ideally around 10) works belonging to, or at least related to, your work. You must describe these works and position your project w.r.t. them (i.e., clearly identify the similarities and differences between your project and each of these works). Also remember if you find that you are detailing topics that you have not introduced already here you need to add something to the earlier Scope section. % Background Theory 

\chapter{DynamiCrypt}
\label{chap:problem}
\lhead{\emph{Problem Statement}}
%The key question to be addressed in this chapter is: "What do I want to achieve".

%This chapter should comprise around 1500 words and describe the problem you are trying to solve. Try to be as specific here as you can, this will help you to anticipate possible risks such as lack of support from APIs.

\section{Problem Definition}
%Describe the problem you are trying to solve in this project. There will sometimes be a need at some point during the report to display an equation that may be core to your project. For example if the project is on gait detection what equation are you using to determine gait? If the project is on localization what is the method/formula? The formatting of these is reliably done in Latex also as we can see in equation \ref{eq:Legrange}.

%\begin{equation}
%\frac{d}{dt}(\frac{\partial L}{\partial \dot{c_i}})-\frac{\partial L}{\partial %c_i}+\frac{\partial P}{\partial \dot{c_i}} = F_i,
%\label{eq:Legrange}
%\end{equation}
The problem this project is designed to solve is increase the level of security when transferring data between hosts. Traditionally when transferring data between two hosts an encryption key for that session is generated and normally used until the session ends. In some cases the same key is used for future communications. This can cause issues and potential data theft if the attacker manages to capture the session and get access to the key by any means necessary, it would be possible to decrypt the session and potentially reveal important/private information.

The information revealed can potentially cause data breaches. In 2018 alone major data breaches have occurred leaking millions of private user data including Facebook where 87 million users were compromised \cite{Data_breach_facebook}. The largest data breach this year affected 1.1 billion registered Indian citizens. The company that was compromised is Aadhaar \cite{Data_breach_Aadhaar} causing personal addresses, phone numbers, emails and names to be available to anyone willing to part with a few hundred rupees.

This project will mitigate this potential problem by periodically changing the encryption key throughout the duration of the session with both parties being aware of the current key in use while the attacker will not know the key. By using this method if the attacker manages to get the encryption key only a small portion of the session could be decrypted. The random keys generated for encryption will be isolated from each other therefore if the attacker has one key it will not be possible to perform mathematical operations on the key to calculate the next keys used for encryption. 

\section{Objectives}
%Enumerate the objectives you want to achieve in your project. Again as this is an early stage these will tend to change but there should be a rational explanation for this change. Always document your work, keep a lab book during the term that you only use for FYP!
The objectives of this project focus on providing a sort of an API that allows to dynamically synchronises with other remote hosts and shares encryption keys with other servers running on the same machine, in order to provide dynamically encrypted information.

Objectives are identified in a way that makes then achievable under the time constraint.

\begin{enumerate}
	\item Provide an API that can be configured only by servers running only on the same machine.
	\item Provide a number of tree parity machines that will synchronise with remote tree parity machines these will be configured through the API and will be able to access machines outside of localhost.
	\item Provide methods of identification for threads, hosts and threads for the different hosts.
	\item Provide methods for allowing multiple hosts to access the API.
	\item Evaluate performance and maybe use and test different learning algorithms.
	\item Create an easy to use NodeJs module for easy implementation with NodeJs Apps.
\end{enumerate}

\section{Functional Requirements}
%Enumerate the functional requirements you want your project to have. 

%Please, do not include the use cases here. If you want to create a one-to-one mapping between functional requirements and use cases (which does not necessarily need to be the case, indeed most likely this will not be the case) do it elsewhere. Here should purely describe what do you want to do. In no case should you use this section to provide a description of how to implement them, that is for later. For people doing projects that are not heavy implementation projects (e.g. deploying an architecture or testing a novel tool in specific conditions) this structure can still be used as it will force you to think about what you plan to achieve and what possible metrics you may need to measure success.

%Let me explain this with more detail. A common mistake is that people confuse the problem description with the solution approach. This is a common mistake by confusing the \emph{what} with the \emph{how}. Here we are purely focused on the what: What is this project about? What are the objectives? What are the functional and non-functional requirements? 

%How are we going to do all these things? Well, this is a question for next chapter. Provided a problem, an objective or a functional requirement, obviously there will usually be many ways of doing it, thus there will be many \emph{hows}, but the definition, the \emph{what} we want to achieve will be unique.

%One other display structure you may wish to use at some stage during the report is a figure array. This can also be easily done with Latex and is shown in figure \ref{fig:twosuccesskid}

%\begin{figure}
%\centering     %%% not \center
%\subfigure[Figure A]{\label{fig:a}\includegraphics[width=0.48\textwidth]{successkid.jpg}}
%\subfigure[Figure B]{\label{fig:b}\includegraphics[width=0.48\textwidth]{successkid.jpg}}
%\caption{Two Success kids}
%\label{fig:twosuccesskid}
%\end{figure}
\begin{enumerate}

	\item Share connection information between hosts. Port number(s), number of synchronising threads, thread id(s) and maybe more.
	\item Threads should be easily identified as there will be multiple threads synchronising simultaneously. 
	\item Each thread synchronises with remote thread.
	\item API sends one key to another server running on localhost and only servers running on localhost after synchronisation between one or more threads occurred.
	\item After key is extracted from thread the localhost thread and corresponding remote thread will desynchronise and begin to synchronise again.
	\item API will be notified when a host disconnects and connects to clean up accordingly.
	\item sensitive information in ram should be protected/encrypted.
	\item sensitive information will not be saved in logs or cache.
	
\end{enumerate}




\section{Non-Functional Requirements}
%Enumerate the non-functional requirements you want to achieve in your project (i.e. broadly speaking how your system will operate).
\begin{enumerate}

	\item The first thread to synchronise should take no longer than 2 seconds.
	\item The program should work on any Linux host with a kernel version of at least 3.10.
	\item The NodeJs module should support NodeJs version 8 and above. 
	\item The NodeJs module should provide easy to use callbacks and hide the complexity.
	\item The install script should work on all supported distros provided dependencies are met. 
	\item The API should require authentication.
	\item The synchronising threads should have some sort of digital signature or a method of identification.
	\item The API should support the ability to synchronise with multiple hosts at the same time or multiple instances of the API can be executed or some form of master deals with multiple instances. Perhaps a different port can be used for each connecting host.
	\item Program should run under its own user to minimise tampering.
\end{enumerate}


 % Problem

\chapter{Implementation Approach}
\label{chap:implementation}
\lhead{\emph{Implementation Approach}}

%The key question to be addressed in this chapter is: "How do I plan to achieve what I have outlined in the previous chapter".

%This chapter should comprise around 5000 words and specify your planned implementation approach. Again all sections below are suggestions and will vary significantly from project to project, the key element to be addressed is the core question of the chapter.

\section{Architecture} \label{sec:Arch}
%Describe the architecture of the solution that you have in mind, including:
%\begin{itemize}
%    \item Technologies involved (e.g., frameworks, programming language). 
%    \item The hardware needed to develop the project (and to support at deployment stage)
%\end{itemize}

%Provide a high level view of the system you have in mind, including any package of classes, what is it responsible for and what other packages it communicates to. Provide a high level view of the database (or structure) needed to support the project, including what each table/document is responsible for and the hierarchy among them. You need to be as specific here as you can, why? Because this will aid you in identifying parts of the project you are vague on, this may be fine for some components but cause problems in term 2 for others. If you have hardware element in your project this is also where you provide a high level view of how these elements integrate into the project. So for a project that is cyber-physical you will have both a hardware and software architectural diagram. N.B. This is NOT a full system design but a high level overview of what you can credibly develop. This architecture should be informed by prototyping activity. 

%Some of the implementation focused projects may describe how do you envision tackling the functional requirements of your project via a set of use-cases. DFDs are also helpful here to understand elements of your project that may cause problems. You should describe the role of the different parts of the architecture of the solution, and the interaction among them.

Designing a software architecture is a crucial step of creating software. This will attempt to illuminate any uncertainties that come up with the functionality or implementation. Planning out the architecture before coding often results in a more stable and scalable software. This section will be dedicated to explaining how the project is going to work. The technologies and languages used for this project will be introduced as needed. High level diagrams and class diagrams will accompany the explanation, these diagrams will most likely change during the implementation phase as it would be rather amazing if I got every thing correct here.

The project will be written in C++ as it is a binary compiled language resulting in fast execution of code, this is required as synchronisation takes up a reasonable amount of execution time. C++ has libraries for dealing with networking and threads therefore building an API will be no problem. I also plan on doing a NodeJs module for easy integration with the API using any NodeJS application, this will be done in JavaScript.

\begin{figure}[!h]
  \centering
      \includegraphics[width=1\textwidth]{Figures/basic-2hosts.jpg}
  \caption[Two hosts using Dynamic Crypt System]{Two hosts using Dynamic Crypt System}
  \label{fig:2hostsbasic}
\end{figure}

\FloatBarrier

The diagram \ref{fig:2hostsbasic} shows a very high level overview of the system in action. There are two hosts in this example. Each host has a localhost Application which is any application that uses the API, it can be a python web application, NodeJS web application, PHP web application etc... 

Each host has a Dynamic Crypt software running which is what this project is about. 

The Dynamic Crypt has two main sort of components if you like. The API which is responsible for communication between applications running on localhost only for security purposes. The localhost applications communicate with the API through a port that will be configured with iptables to only be available for local addressees only so 127.0.0.1 only. 

The Synchronising thread "component" is a single Tree Parity Machine as well as the necessary functionality required for networking and synchronisation between the partner thread of the other host. Each synchronising thread will have its own unique id as well as the partner thread id, IP, port etc.. from the other host in order to be able to send packets directed at the partner thread. This is needed because there will be multiple instances or threads for each host connected to a host. In the host 1 to host 2 example there will most likely be only ten threads synchronising with each other this is to ensure a steady supply of encryption keys, this number of threads for each host may change during the implementation phase. 

How all of this works is going to be similar to the following. 

Host 1: Localhost application establishes a connection with Host 2: Localhost application using standard methods. 

When local Host 1: Localhost application wants to use dynamic encryption to send data to Host 2: Localhost application it contacts the Host 1: Dynamic Crypt API with an init request. 

Host 1: Dynamic Crypt API takes note of Host 1: Localhost application details such as the applications name or port number to distinguish the application from other local host applications as will be demonstrated in a later example. and initialises ten Synchronising threads. The API associates this list of synchronising threads with the application and sends back the details of each thread, the port the threads will use and the IP of Host 1 to Host 1: Localhost application.

Host 1: Localhost application then essentially forwards this information to Host 2: Localhost application which will know that this is an dynamic synchronisation init request and will forward that information to Host 2: Dynamic Crypt API through the Host 2: Dynamic Crypt API port. The Host 2: Dynamic Crypt API will initialise ten Synchronising threads and assigns a partner to them which is essentially the partner id of one of the Host 1: Dynamic Crypt threads.

Host 2: Dynamic Crypt threads will send a request to Host 1: Dynamic Crypt threads. The Thread manager will then forward the info to each thread assigning the partners thread id for Host 1 threads. Now synchronisation between threads will occur how synchronisation occurs is discussed in the research phase but I will provide a diagram shortly.

When a thread is synchronised the API of both hosts will notify the corresponding Localhost application that dynamic synchronisation may occur. The key of the synchronised thread is saved in the API and the thread is forced to desynchronise and begin synchronisation again in order to get a new key. These keys will be added in a queue on both of the hosts API.

The Localhost application can now call the API with an encrypt message request and pass a message to be encrypted. The API uses a key to encrypt a message then the key is discarded from the API, the encrypted message is returned to the Localhost application for ready for sending. 

The message arrives at the other Localhost application. The Localhost application will send a decrypt request to the API with the encrypted message. The API will use the appropriate key to decrypt the message and send it back to the Localhost application. 









\begin{figure}[!h]
  \centering
      \includegraphics[width=1\textwidth]{Figures/basic-3hosts.jpg}
  \caption[Host one using Dynamic Crypt System with host two and three at the same time]{Host one using Dynamic Crypt System with host two and three at the same time}
  \label{fig:3hostsbasic}
\end{figure}

\FloatBarrier

The diagram \ref{fig:3hostsbasic} demonstrates how dynamic encryption can be implemented when Host 1: Localhost application wishes to send dynamically encrypted information between Host 2 and Host 3 simultaneously. Because Host 1 is attempting to use dynamic encryption to communicate between Host 2 and Host 3 only this means that Host 2 and Host 3 are not aware of each other as there is no need for them to communicate.

This works in quite a similar fashion as with only two hosts but now there are three.

Host 1: Localhost application establishes a connection with Host 2: Localhost application and Host 1: Localhost application establishes a connection with Host 3: Localhost application using standard methods more than likely this will happen one after the other as extra data might be required from Host 3, however there is not going to cause any problems doing it at the same time. 

Host 1: Localhost contacts the Host 1: Dynamic Crypt API with an init request twice one for Host 2 and Host 3. In the init request Host 1: Localhost application also provides any name for Host 2 and Host 3 to distinguish quickly between the two later on.

Host 1: Dynamic Crypt API takes note of Host 1: Localhost application details such as the applications name or port number and the name given to Host 2 and Host 3, and initialises ten Synchronising threads for Host 2 and another ten Synchronising threads for host 3. The API associates this list of synchronising threads with the application and sends back the details of each thread, the port the threads will use, the IP of Host 1 and lastly the name given to Host 2 and Host 3 to Host 1: Localhost application. These details are sent back for Host 2 and Host 3 separately to allow for easy growth since Host 1 might want to dynamically encrypt information and send it to Host 3 much later than Host 2.

Host 1: Localhost application then essentially forwards this information to Host 2: Localhost application and Host 3: Localhost application these will know that this is an dynamic synchronisation init request and will forward that information to Host 2: Dynamic Crypt API through the Host 2: Dynamic Crypt API port and Host 3: Dynamic Crypt API through the Host 3: Dynamic Crypt API port. The Host 2: Dynamic Crypt API and Host 3: Dynamic Crypt API will each initialise ten Synchronising threads and assign a partner to them which is essentially the partner id of one of the Host 1: Dynamic Crypt threads.

Host 2: Dynamic Crypt threads will send a request to Host 1: Dynamic Crypt threads followed by another request from Host 3: Dynamic Crypt threads to Host 1: Dynamic Crypt threads. The Thread manager will then forward the info to each thread assigning the partners thread id for Host 1 threads. Now synchronisation between threads will occur.

The last three steps are identical as in the last example.



\begin{figure}[!h]
  \centering
      \includegraphics[width=1\textwidth]{Figures/basic-3hosts-2apps.jpg}
  \caption[Host one using Dynamic Crypt System with host two and three with different applications using the API]{Host one using Dynamic Crypt System with host two and three with different applications using the API}
  \label{fig:3hostsbasic2apps}
\end{figure}

\FloatBarrier

The diagram \ref{fig:3hostsbasic2apps} is quite similar to the last example where as in this case there are two local host applications running on Host 1: In this case localhost application 1 is using the dynamic encryption system to send data to Host 2 and localhost application 2 is using the dynamic encryption system to send data to Host 3. They both use the same API and therefore this example is quite similar to the previous one.

Host 1: Localhost application 1 establishes a connection with Host 2: Localhost application using standard methods.
Similarly Host 1: Localhost application 2 establishes a connection with Host 3: Localhost application using standard methods.

Host 1: Localhost application 1 contacts the Host 1: Dynamic Crypt API with an init request. 
Host 1: Localhost application 2 contacts the Host 1: Dynamic Crypt API with an init request. 

Host 1: Dynamic Crypt API takes note of Host 1: Localhost application 1 details such as the applications name or something to distinguish it from Host 1: Localhost application 2, and initialises ten Synchronising threads. The API associates this list of synchronising threads with the application and sends back the details of each thread, the port the threads will use and the IP of Host 1 to Host 1: Localhost application 1 and similar but different values are sent to Host 1: Localhost application 2 as well. 

Host 1: Localhost application 1 forwards this information to Host 2: Localhost application which will know that this is an dynamic synchronisation init request and will forward that information to Host 2: Dynamic Crypt API through the Host 2: Dynamic Crypt API port. 
Host 1: Localhost application 2 forwards this information to Host 3: Localhost application which will know that this is an dynamic synchronisation init request and will forward that information to Host 3: Dynamic Crypt API through the Host 2: Dynamic Crypt API port. 
The Host 2: Dynamic Crypt API and The Host 3: Dynamic Crypt API will initialise ten Synchronising threads and assigns a partner to them which is essentially the partner id of one of the Host 1: Dynamic Crypt threads.

Host 2: Dynamic Crypt threads will send a request to Host 1: Dynamic Crypt threads followed by another request from Host 3: Dynamic Crypt threads to Host 1: Dynamic Crypt threads. The Thread manager will then forward the info to each thread assigning the partners thread id for Host 1 threads. Now synchronisation between threads will occur.

The last three steps are identical as in the first example.



From the three use case examples the API must be able to support:
\begin{itemize}
    \item A single localhost application connecting with another dynamic encryption system.
    \item A single localhost application connecting with multiple other dynamic encryption systems.
    \item Multiple localhost applications connecting with other dynamic encryption systems.
    \item Multiple localhost applications connecting with multiple other dynamic encryption systems per each local host application.
\end{itemize}

\begin{figure}[!h]
  \centering
      \includegraphics[width=1\textwidth]{Figures/cppframeworks.png}
  \caption[Comparison of different C++ rest frameworks]{Comparison of different C++ rest frameworks\cite{Crestframeworks}}
  \label{fig:Crestframeworks}
\end{figure}

\FloatBarrier

In order to build a proper API I will use a framework as making one from scratch would be too time consuming. There are not many C++ only rest APIs frameworks so the choice was not too difficult. 
Table \ref{fig:Crestframeworks} shows the most popular C++ rest frameworks. Cpprestsdk is made by Microsoft and is generally the most popular one to use. However I have tried installing the library and had issues compiling the examples provided on their GitHub a simple hello world server worked perfectly but the rest API example had issues. This is perhaps for the better that I was unable to get it to work and had to find other frameworks to work with. I stumbled upon the websites that the table is from and the framework Pistache \cite{pistache} seems to be leaps ahead of Cpprestsdk having a low request time of only 6 milliseconds and a large 320 requests per second capability which is what I need for this project and considering that it will take quite a lot of requests to synchronise tree parity machines. Fortunately I was able to install the library correctly and the example rest API code for Pistache compiled and worked correctly.

Therefore I will be using Pistache as the rest API part of this project and the tree parity part of the code also since I wanted my software to utilise two ports.



\begin{figure}[!h]
  \centering
      \includegraphics[width=1\textwidth]{Figures/API.jpg}
  \caption[Rest API class diagram]{Rest API class diagram}
  \label{fig:myapi}
\end{figure}

\FloatBarrier

The class diagram \ref{fig:Crestframeworks} demonstrates roughly how the API will look like. The variables at the top of the DynamicServer class are mostly used for the library. serverLock is to prevent issues with threads changing the same variable at the same time. metrics is for measuring various server performances this uses the Metric class defined in the diagram. httpEndpoint is a pointer used by the framework internally. router is used for handling different routes in the setupRoutes function. keys is a struct array that has the generated encryption key from the tree parity machine, the id of the tree parity machine and a status whether it was sent of or not more properties will most likely be added in the implementation phase to this struct.

The functions after setupRoutes are all route handlers that will need to be associated with a route in the setupRoutes function. initCrypt function will tell the tree parity machine handler to setup the tree parity machines. receiveNewDynamicCryptMachine is when a different host setup the tree parity machines and sends you ids and other information to allow for partnering of this hosts tree parity machines. sendNewDynamicCryptMachine is the opposite of the previous this time the api is sending info of its tree parity machines to another host. stopDynamicCryptMachine this is used when the application believes it doesnt need any more new keys for dynamic encryption so the synchronisation will cease to avoid expensive unnecessary operations. getNextKey will send the next if available key to the application. restartDynamicCryptMachine would be typically called after stopDynamicCryptMachine to generate more keys once again. goodBye will be called when the application doesn't need to use the API anymore this will remove any references and any tree parity machines relating to the application. When the application disconnects from the API a similiar set of instructions will also occur. auth is when authorisation will be implemented for increased security. handleReady is more of a test function to determine if server is up and running. printCookies will most likely never be used. 

The Metric class is used for saving different server metrics. A metric has a name, value and increment amount.

Lastly the DynamicApi class holds the main method that initialises the server.





\begin{figure}[!h]
  \centering
      \includegraphics[width=1\textwidth]{Figures/TreeParityMachine.jpg}
  \caption[Tree Parity Machine synchronisation class diagram]{Tree Parity Machine synchronisation class diagram}
  \label{fig:tpmClassDiagram}
\end{figure}

\FloatBarrier

this is reference to diagram\ref{fig:tpmClassDiagram}.







\section{Risk Assessment}
%Identify any potential risk precluding you from successfully complete your project. This section is really important and often neglected by students resulting in fatal risks occurring in some projects. Make sure to give this section the time it requires. Classify the risk according to their importance, possibility of arising and enumerate the decisions you can make to anticipate them or mitigate them (in case they finally arise). Table \ref{tab:ProjRisks} may help with this classification. This section should include your mitigation approach for any critical risks.

%\begin{table}[h]
%\centering
%\scriptsize
%\caption{Initial risk matrix}
%\begin{tabular}{|p{2cm}|p{2cm}|p{2cm}| p{2cm} |p{2cm}| p{2cm}|}
%\hline \bf Frequency/ Consequence & \bf 1-Rare & \bf 2-Remote & \bf 3-Occasional & \bf 4-Probable & \bf 5-Frequent\\ [10pt]

%\hline \bf 4-Fatal & \cellcolor{yellow!50} & \cellcolor{red!50} & \cellcolor{red!50} & \cellcolor{red!50} &\cellcolor{red!50} \\ [10pt]

%\hline \bf 3-Critical &\cellcolor{green!50} & \cellcolor{yellow!50} & \cellcolor{yellow!50} & \cellcolor{red!50} &\cellcolor{red!50} \\ [10pt]

%\hline \bf 2-Major & \cellcolor{green!50} & \cellcolor{green!50} & \cellcolor{yellow!50} &\cellcolor{yellow!50} &\cellcolor{red!50} \\ [10pt]

%\hline \bf 1-Minor & \cellcolor{green!50} & \cellcolor{green!50} & \cellcolor{green!50} &\cellcolor{yellow!50} &\cellcolor{yellow!50} \\ [10pt]
%\hline
%\end{tabular} \\
%\label{tab:ProjRisks}
%\end{table}

\section{Methodology}
%Describe your personal approach on how to tackle the different parts of this project, including:
%\begin{itemize}
%    \item How to tackle the needed research to fulfill the background chapter. 
%    \item How to set up your Computer Science skills to the project needs (e.g., describe your plan to learn any new technology involved on the project that you are not familiar with). 
%    \item What core project managing approach will you follow (e.g., Waterfall, Scrum, etc).
%\end{itemize}

\section{Implementation Plan Schedule}
%Come up with a schedule for the remaining time (including second semester), so as to describe how do you envision to achieve the implementation of your project by the end of semester 2. This plan SHOULD be ambitious but MUST be realistic and SHOULD be informed by early prototyping and MUST be discussed with your term 1 supervisor.

\section{Evaluation}
%Come up with an evaluation plan that allows you to measure how much have you actually achieved the goals of your project. This again is a section that is often neglected where students loose marks. How do you plan to measure the output of your project? A binary it works/does not work is insufficient. You need to be able to quantify the success against both the functional requirements and the initial idea. These are not the same as you may meet all function requirements outlined but not solve the overall problem because you have failed to revisit these and update them with new information which you learn as you are developing the project.

\section{Prototype}
%Although you do not have a fully functional project yet, you should show wireframes, snapshots or representation on how do you envision your project to look once the implementation phase has been completed. The nature of this section will vary significantly from project to project and can include anything from code snippets to snapshots of service deployments. Any prototyping you have done during the term should be summarized here that has not been captured in earlier sections. For example if you are planning to host your project using AWS in an EC2 instance you should have at least created a "hello world" setup to determine the basics, this probably should have been discussed in section \ref{sec:Arch}. % Solution Approach

\chapter{Implementation}
\label{chap:imp}
\lhead{\emph{Project Implementation}}
%This chapter should comprise 15 pages and enumerate your experience when doing what you wanted to do the way you wanted to do it.
The implementation of the system has changed drastically since the original plan due to an oversight where it was believed that the Pistache framework could handle the functionality for both the API and the synchronisation between the tree parity machines. The project turned out to be much more involved and difficult than originally expected. This means that the original sprint plan is no longer representative of the actual sprint plan used. This is however to be expected in an agile environment where the following sprint is normally determined after the evaluation of a sprint that was just completed. Despite all of this the system is functional and produces the same outcome as intended.

\section{Difficulties Encountered}
%Enumerate the different difficulties you have found when developing your solution approach. Create three categories of difficulties:
%\begin{itemize}
%    \item \textbf{Easy}: You managed to solve the problem with little difficulty.
%    \item \textbf{Medium}: It was not easy to solve, but you managed to develop a workaround or solution and %still achieve the functionality you originally had in mind.
%    \item \textbf{Hard}: The difficulty was so complicated that you didn’t managed to solve it. As a result, some functional requirement / non-functional requirement or use case from your solution approach was not achieved.
%\end{itemize}

%For each difficulty, classify it into easy, medium or hard. Then, provide the following info:
%\begin{enumerate}
%    \item Description of the difficulty: Brief description of the problem you found.
%    \item How did it affect the original project design?: Indicate how this difficulty affected:
%    \begin{enumerate}
%        \item the architecture of your solution
%        \item if it represented a risk to your project
%        \item if it affected your methodology to develop your project
%        \item if it changed your implementation schedule
%        \item if it changed the evaluation plan
%    \item What did you do to manage the difficulty arisen?: Brief description of your decision to overcome the %difficulty.
%    \end{enumerate}
%\end{enumerate}
The major difficulties I encountered were primarily related to the architecture of the project. The initial class diagrams were way too basic and didn't account for the fact that Pistache would now only be used for the API and nothing else. 

\textbf{Easy Difficulty}
\begin{enumerate}
\item The NodeJs module was replaced with simply a NodeJs App. The NodeJS module is not implemented due to a time constraint and is not necessary to the outcome of the project. The NodeJs module was originally meant to simplify the communication with the API for people who wish to use it. A NodeJS App would need to be implemented regardless in order to use that module and demonstrate the usability of the API. 
The API doesn't have many routes to use therefore it would not be difficult to use the API with any NodeJs App as interacting with the API is simply done with POST requests. The final NodeJs App that is in the same GitHub repository, consumes the API perfectly as well as purposely demonstrating some aspects of dynamic cryptography that would normally be hidden. Any user that wishes to use the API can simply copy the functions in the NodeJs App that interact with the API and change them accordingly to send and receive their own data. It is recommended for users to copy said functions as there are specific steps that need to take place to register the NodeJS App with the API in particular the synchronisation service. For this reason The NodeJs App uses pure NodeJs libraries like ExpressJs \cite{ExpressJS} which is the most popular routing framework for NodeJs. The NodeJs App has no front end as it would be difficult for users wishing to use the API to adapt the front end too, therefore only a basic HTML template engine is used to deal with the HTML. JavaScript for the front end is purely used for visual purposes and is not required since all of the information sent from the browser to the NodeJs app is done through classic HTML forms. This doesn't have any impact on the architecture as the NodeJS App is not really included in the architecture as it is simply designed to consume the API. This does impact the schedule in a positive way as it is easier and quicker to simply make a NodeJS App than a NodeJs module and a NodeJs App.
\end{enumerate}

\textbf{Medium Difficulty}
\begin{enumerate}
\item The sprints needed to be adjusted to accommodate the new implementation plan. The final sprint plan will be placed here as it briefly reflects the major changes that took place, these changes will be discussed in more detail in the following sections. The Sprint plan follows a two week sprint approach.
\begin{table}[h]
\centering
\caption{New Sprint Plan}
\begin{tabular}{|p{1cm}|p{12cm}|}
\hline Sprint & Tasks \\ [14pt]

\hline 1 & Create a basic API server that can process GET and POST requests. Familiarise self with RapidJSON library \cite{rapidjson} and create functions to parse C++ objects into JSON "strings" and JSON "strings" back into C++ objects. Research peer to peer libraries, no useful ones were found. Decided to make a custom one with the help of the Boost ASIO \cite{boost_asio_home} library. Never used Boost ASIO before therefore a decision was made to follow the developers tutorials on the Boost ASIO website. This was not enough to understand the complicated library therefore other tutorials such as this one \cite{boost_asio_totorial_1} were also followed.  \\ [12pt]

\hline 2 & Further knowledge of Boost ASIO needed to be acquired before it was possible to proceed with coding the actual peer to peer network. Thankfully a very informative book called Boost.Asio C++ Network Programming \cite{boost_book} by John Torjo was purchased and ended up being the last resource needed in order to complete the peer to peer network. A basic peer to peer network was written however was unstable at the end of this sprint \\ [12pt]

\hline 3 & Peer to peer network works as expected. Implemented basic synchronisation with real tree parity machines. Implemented command line parsing using Boost Program Options \cite{boost_asio_cmd} to enable different options such as selective outputs and port configurations. Completed synchronisation between tree parity machines fully therefore they produce the same weights successfully. Implemented "logging" to external terminal windows. \\ [12pt]

\hline 4 & Research how to proceed with the API i.e have two separate processes or contain the API and peer to peer network within the same process. Decision was made to contain both in the same process for security reasons. Basic API server was constructed and could be easily spawned along side the peer to peer service.\\ [12pt]

\hline 5 & Implement API functionality for synchronisation. Changes needed to be made to the peer to peer service to accommodate unexpected requirements needed to begin synchronisation when using the API. AES encryption decryption implemented using Crypto++ \cite{cryptopp} library. Encrypted data needed to be encoded using base64 again with the help of Crypto++ library. Implement a NodeJs app to consume and demonstrate various features of the API.\\ [12pt]

\hline 5 & Test the system. Implement options to choose which parts of the system to log since logging everything at once can be difficult to demonstrate different aspects of the system. Clean up the code. Document instructions on how to setup the system in the GitHub Readme file. \\ [12pt]

\hline
\end{tabular} \\
\label{tab:ProjRisks}
\end{table}

\FloatBarrier


\item The API and the synchronisation could not be done using only the Pistache framework as originally believed. This was a fairly significant set back since I have experimented before with Pistache framework and was familiar with how it operated functionality wise. It was only until I attempted to use Pistache for synchronisation I discovered that my original plan will not work since Pistache is a very good REST framework and not much else. To synchronise the tree parity machines a peer to peer network seemed most appropriate in how the synchronisation was envisioned to take place. After searching for easy to use libraries that supported peer to peer networks nothing that matched the requirements was found. This lead to the conclusion that a custom peer to peer network would have to be built from scratch. The attention grew towards Boost Asio as it is the most flexible generic networking library for C++. Unfortunately this library is quite extensive and is not as abstract as some of the previous peer to peer libraries encountered. Since with Boost Asio you will still deal with sockets however this will allow for greater control provided you know how the library works and Boost Asio has excellent asynchronous support which is essential for a responsive server. It took almost two full sprints to learn how the library works which set my initial schedule back as it was planned to spend time to learn the new technologies that would be used however it was not expected to spend so much time learning said new technologies. However some of the lost time was made back since the design of the peer to peer network was simple and efficient which allowed for a large number of synchronisation requests to take place per second resulting in generating a valid key in around one second which satisfies the requirement and no further optimisations needed to take place. This occurrence changed the architecture significantly since originally this wasn't meant to be present. Overall I'm glad that this happened as I had a chance to learn one of C++'s most popular networking library. Another benefit of using Boost Asio for synchronisation over Pistache apart from the fact that its impossible to achieve using Pistache is because the Boost Asio library is a lot lighter the performance increase is rather large. Obviously this would be impossible to test fairly however based on Pistache stress tests it can manage roughly 300 - 400 requests per second. The peer to peer Boost Asio can process around one thousand requests per second as it takes around one thousand requests for the tree parity machines to synchronise and they synchronise generally once per second.

\end{enumerate}


\textbf{Hard Difficulty}
\begin{enumerate}
\item A major difficulty that has prevented the implementation of a non functional requirement was the ability to use multiple tree parity machines per peer. Originally the plan was to implement multiple tree parity machines per peer, however in practice after spending more than a week on the problem a decision was made to move on and only allow one tree parity machine per peer. This theoretically does not have any impact on performance and certainly has no impact on security. A single tree parity machine per communication works as expected, a peer can have multiple tree parity machines where for example peer1 communicates with peer2 and peer3 at the same time. In this scenario peer1 would have two tree parity machines as the peer will communicate with two other peers. This scenario works flawlessly in the final implementation of the project. If the original plan were to be implemented peer1 would have twenty tree parity machines, ten per peer it communicates to. The reason why this is difficult is because the ten tree parity machines are reading and writing to the same socket. Whereas with one tree parity machine it has exclusive read and write access to the socket. An attempt was made to implement multiple read and write buffers, a queue, custom read and write objects however a decision was made to abandon this feature for now as it took up too much time and didn't add anything beneficial to the project apart from "being cool". 
The architecture of the project can be said to have changed although it is insignificant as the number of tree parity machines is rather virtual in nature. This issue did not represent any major risk to the project as everything works perfectly with one tree parity machine per connection. The implementation schedule was affected as a lot of time was spent on a feature that did not get implemented and it is better to spend time on features that actually do get implemented.
\end{enumerate}



\section{Actual Solution Approach}
In December of last year, when writing the first version of the report, in Chapter 4 you came up with your original solution approach. On it, you presented (i) the architecture of your solution, (ii) your list of use cases, (iii) a risk assessment, (iv) a methodology to develop your solution approach, (v) your implementation schedule, (vi) your evaluation plan and (vii) some prototype of the resulting product. From January to April you have been developing your solution approach. Along the way you have encountered difficulties (the ones listed in Section 5.1) which might have modified your original plan so that you can come up with an actual developed project.

This section is effectively the production of "as built" specification where you compare your original design to the final finished project. Please go section by section (the ones listed from (i) to (vii) in the last paragraph. For each section, enumerate any difference between the original design and the final project, and justify the difficulty forcing you to make such this change. Do not fret if some of these changes are radical, what is important here is that there is a clear rationale for changes made.

\begin{figure}[!h]
  \centering
      \includegraphics[width=1\textwidth]{Figures/basic-2hosts-final.png}
  \caption[High level overview of architecture]{High level overview of architecture}
  \label{fig:basic-two-hosts-final}
\end{figure}
\FloatBarrier

\begin{figure}[!h]
  \centering
      \includegraphics[width=1\textwidth]{Figures/class-diagram-all-croped.png}
  \caption[DynamiCrypt Class Diagram]{DynamiCrypt Class Diagram}
  \label{fig:class-diagram-all}
\end{figure}
\FloatBarrier

\begin{figure}[!h]
  \centering
      \includegraphics[width=1\textwidth]{Figures/API_class_Diagram-croped.png}
  \caption[DynamiCrypt API Class Diagram]{DynamiCrypt API Class Diagram}
  \label{fig:API_class_Diagram}
\end{figure}
\FloatBarrier

\begin{figure}[!h]
  \centering
      \includegraphics[width=1\textwidth]{Figures/class_diagram_dynamicrypt-part1.png}
  \caption[DynamiCrypt Class Diagram part 1]{DynamiCrypt Class Diagram part 1}
  \label{fig:class_diagram_dynamicrypt-part1}
\end{figure}
\FloatBarrier

\begin{figure}[!h]
  \centering
      \includegraphics[width=1\textwidth]{Figures/class_diagram_dynamicrypt-part2-croped.png}
  \caption[DynamiCrypt Class Diagram part 2]{DynamiCrypt Class Diagram part 2}
  \label{fig:class_diagram_dynamicrypt-part2}
\end{figure}
\FloatBarrier

\begin{figure}[!h]
  \centering
      \includegraphics[width=1\textwidth]{Figures/nodeJsApp.png}
  \caption[NodeJs App architecture]{NodeJs App architecture}
  \label{fig:nodeJsApp_architecture}
\end{figure}
\FloatBarrier % Conclusions and Term 2 work

%% ----------------------------------------------------------------
\label{Bibliography}
\bibliographystyle{IEEEtranN}  % Use the "IEEE Transaction" BibTeX style for formatting the Bibliography
\bibliography{Bibliography}  % The references (bibliography) information are stored in the file named "Bibliography.bib"
\lhead{\emph{Bibliography}}  % Change the left side page header to "Bibliography"

%% ----------------------------------------------------------------
% Now begin the Appendices, including them as separate files

\addtocontents{toc}{\vspace{2em}} % Add a gap in the Contents, for aesthetics

\appendix % Cue to tell LaTeX that the following 'chapters' are Appendices

\input{Appendices/AppendixA}	% Appendix Title

\input{Appendices/AppendixB} % Appendix Title

%\input{Appendices/AppendixC} % Appendix Title

\addtocontents{toc}{\vspace{2em}}  % Add a gap in the Contents, for aesthetics
\backmatter
\end{document}  % The End
%% ----------------------------------------------------------------